\fancyhead[LH]{上海交通大学学位论文}
\fancyhead[RH]{第二章\quad 正文文字格式}
\section{正文文字格式}
\subsection{论文正文}
论文正文是主体,一般由标题、文字叙述、图、表格和公式等部分构成。一般可包括理论分析、计算方法、实验装置和测试方法,经过整理加工的实验结果分析和讨论,与理论计算结果的比较以及本研究方法与已有研究方法的比较等,因学科性质不同可有所变化。\par
论文内容一般应由十个主要部分组成,依次为:⒈封面,⒉中文摘要,⒊英文摘要,⒋目录,⒌符号说明,⒍论文正文,⒎参考文献,⒏附录,⒐致谢,⒑攻读学位期间发表的学术论文目录。\par
以上各部分独立为一部分,每部分应从新的一页开始,且纸质论文应装订在论文的右侧。\par
\subsection{字数要求}
\subsubsection{本科论文字数要求}
各学科和学院自定。理工科研究类论文一般不少于2万字,设计类一般不少于1.5万字;医科、文科类论文一般不少于1万字。
\subsection{引用格式}
引用\cite{label1}
\citet{label2}提出 \dots
\citet{label3}提出 \dots

\subsection{本章小结}
本章介绍了……

\clearsection